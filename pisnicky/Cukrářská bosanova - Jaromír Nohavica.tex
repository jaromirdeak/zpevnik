\beginsong{Cukrářská bosanova}[by={Jaromír Nohavica}]
%filter: Mlok
\begin{textblock}{5}(8,-0.5) \gtab{C\hidx{maj7}}{032000:021000} \gtab{C\shrp{}\didx{dim}}{002323:001324} \end{textblock}
\emptyv
\cseq{\[C\hidx{maj7}] \[C\shrp{}\didx{dim}] \[Dm\hidx{7}] \[G\hidx{7}]}
\cl
\num
\chordsoff
Můj přítel snídá sedm kremrolí
a když je spořádá, dá si repete, cukrlátko,
on totiž říká: \uv{Dobré lidi zuby nebolí
a je to paráda, chodit po světě
a mít, mít v ústech sladko.}
\fin\ifchorded\chordsoff\fi
\chor
\chordson
Sláva, cukr a káva a půl litru becherovky,
hurá, hurá, hurá, půjč mi bůra,
útrata dnes dělá čtyři stovky,
všechny cukrářky z celé republiky
na něho dělají slaďounké cukrbliky
a on jim za odměnu zpívá \[A\hidx{7}]zas a znovu
tuhletu cukrářskou bosanovu.
\cl\ifchorded\chordsoff\fi
\num
Můj přítel Karel pije šťávu z bezinek,
říká, že nad ni není,
že je famózní, glukózní, monstrózní, ať si taky dám,
koukej, jak mu roste oblost budoucích maminek
a já mám podezření, že se zakulatí jako míč
a až ho někdo kopne, odkutálí se mi pryč
a já zůstanu sám, úplně sám.
\fin\ifchorded\chordsoff\fi
\repchorus{Sláva, \ldots}
\num
Můj přítel Karel Plíhal už na špičky si nevidí,
postava fortelná se mu zvětšuje, výměra tři ary,
on ale říká, že glycidy jsou pro lidi,
je prý v něm kotelna, ta cukry spaluje,
někdo se zkáruje, někdo se zfetuje
a on jí bonpari, bon, bon, bon, bonpari.
\fin\ifchorded\chordsoff\fi
\repchorus{Sláva, \ldots}
\endsong



