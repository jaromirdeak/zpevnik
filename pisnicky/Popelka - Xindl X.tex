\beginsong{Popelka}[by={Xindl X}]
\emptyv
\chordsoff
 Capo 5
\cl
\chordsoff
\emptyv
\chordson
\cseq{\[E] \[C] \[G] \[D] \[E] \[C] \[D]}\\
\cl
\num
\chordson
Už zase  \[G]skončil flám,
do vodky \[Em]{ džus} ti dám
a pak ti \[C] spočítám nový  ztr\[G]áty.     \[D]
Úsměv si  n\[Am]amaluj, i kdyby  n\[C]ebyl tvůj,
vždyť co ti  z\[G]bejvá, než se mi  smá\[D]t.
\fin
\freev
\chordson
Popelko \[G] mejdanů, co snídá v  žu\[Em]panu
a nechce \[C] tancovat podle \[G] táty. \[D]
Víš dobře  ku\[Am]{dy jde,} tak ať tě  nemi\[C]ne,
víš, že  zá\[G]zrak prej se smí  stát\[D].
\cl
\chor
\chordson
\[Em]Popel a  \[C]hrách,
\[G]{a  hry}\[D]{,  co} se nedají v\[Em]{y hrát} \[C]
ne\[G]{ž  s} velkejma ztrát\[D]{a ma.}
\[Em]Popel a  \[C]hrách,
k\[G]{do } ví,
\[D]{s kým} Popelko půjdeš  spá\[Em]t.      \[C]
Holu\[D]{bi } odlítli, přeber si to sama\[ECG]{,}
\[D]přeber si to sama..   \[ECD]
\cl
\num
Žiješ na  kolejích,
všichni tě  milují,
krom toho,  kdo chybí ti teď  nejvíc.
Ten, co mu  utíkáš,
neboť mu  zazlíváš,
že tě  měl tak trochu moc  rád.
\fin
\freev
Jsi místní  Popelkou, hraješ si  na velkou,
každou noc  vyzkoušíš novej  střevíc.
Než ze sna  procitáš, půlnoc je  odbitá,
kočár  dávno ztratil se  v tmách.
\cl
\repchorus{\emptyspace}
\num
Žiješ na  kolejích,
vlaky už  vodjely
a město  polyká dětský  stíny.
Pod vokny  napad sníh
a lidi v  ulicích,
jsou tak  prázdný, až se chce  řvát.
\fin
\emptyv
Si diva  z divano, sedáš si  na vanu
a smejváš  z pod vočí černý  splíny.
Víš dobře  kudy ne, nebe tě  nemine,
tak proč  bejváš tak nejis tá.
\cl
\repchorus{\emptyspace}
\freev
\chordson
\[D]Přeber si to sama
\cl
\endsong


