\beginsong{Psáno u zpráv}[by={Tomáš Klus}]
\num
\[Cm]Kráčím stezkou stesku,
\[Fm]ráčím být bez potlesku
\[G\shrp{}]{a čím} tělo mé, dnes  v no\[G]{ci zahřejí?}
\fin
\chordsoff
\num
Setkán z dlouhých tradic,
setkám se s těmi co ví víc
ať hlavu mou setnou raději.
\fin
\chor
\chordson
Já  \[Fm]{člen beze} j\[G]menné gene ra\[Cm]ce,
humor \[G\shrp{}]{ mám} rád, jen  k\[G]dyž je bez le gr\[Cm]ace.
Postrá \[G\shrp{}]{dám poko} r\[G]u,
post rád  \[G\shrp{}]{dám za} pár ko ru\[G]n,
po ko\[G\shrp{}]{ runách} až k  m\[G]ístům hono rac\[Cm]{e.}
\cl
\num
Kývám, přikyvuji,
i vám poděkuji drazí pánové
za klid, za úkryt.
\fin
\num
S dlaní nataženou,
zdaním žvanec ženou.
Sbohem Salome! Sejdem se u koryt.
\fin
\chor
Seděli zpití pod vobrazy,
se dělí s těmi, co vzpomněli jak plazit.
Se děly takové věci,
v té zemi zvířecí,
kde moc si střídá dobytek a plazi.
\cl
\emptyv
\chordson
\cseq{\[Cm] \[Fm] \[G\shrp{}] \[G] \[Cm] \[Fm] \[G\shrp{}] \[G] \[G\shrp{}]}\\
\cl
\num
\chordson
\[C\shrp{}m]Verbuj, verbuj ovocnáři,
\[F\shrp{}m]{ač máš} v erbu sladko,
\[A]zrada je ve tvém jméně, lháři, neb \[G\shrp{}]{lež je} tvoji matkou.
\fin
\num
Táhni vykradači hrobů
a shnij si u všech čertů.
Dřív bili jsme se za svobodu a poslouchali Mertu.
\fin
\chor
\chordson
A když  \[A]pustili nás  z\[G\shrp{}]{ klece,}
jdem zpět \[A] před vysoký  p\[G\shrp{}]ece
pecen\[A]{  chleba} vymě\[G\shrp{}]{ níme} za  \[C\shrp{}m]PéCé.
\cl
\chor
\chordson
\[A]{A pak}  \[G\shrp{}]vladaři
\[A]daj povel  \[G\shrp{}]krysaři,
\[A]všichni se  p\[G\shrp{}]ůjdem koupat  k \[C\shrp{}m]řece\ldots{}
\cl
\endsong


