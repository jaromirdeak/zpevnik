\beginsong{Houbová balada}[by={Michal Horák}]
\num
\[Em]{  V jednom} lese \[Am]{  na mechu}
\fin
\chordsoff
\freev
\chordson
\[D] bylo slyšet  \[G] povzdechu, když
\cl
\freev
\chordson
\[C] houba Emil \[A\hidx{7}]  zaklel, že už   \[D]dál nechce bejt   s\[D\hidx{7}]ám.
\cl
\freev
\chordson
\[Em]{  „Tady} končej   f\[Am]óry,
\[D]{  docházej} mi  \[G] spóry
a \[C]{  sporangio}\[A\hidx{7}]{ fóry} už   \[D]namožené   má\[D\hidx{7}]m.
\cl
\freev
\chordson
\[C]{  Náhle}  \[G] před ním   z\[H\hidx{7}]jevila  s\[Em]{e}
\[C]{  houba} s nožkou  \[G] na štíhlé   li\[H\hidx{7}]nii,
\[C]{  oba} \[G]{  náhle}  \[H\hidx{7}]{ pocítili}   \[Em]chemii
a \[C]{  než} se oba  \[G] nadáli už   p\[H\hidx{7}]rováděli … gametangiogamii   .    \[Em]
\cl
\num
Pohodové symbiózy
čas byl velmi krátký,
Emil mohl oči nechat na vedlejší bříze.
Začali si vyměňovat
minerální látky,
došlo mezi nimi k nelegální mykorhize.
\fin
\emptyv
Emilova žena vycítila nevěru
a každá hyfa v těle se jí zježila,
k smrti utrápena jednou takhle k večeru
načapala v tom nejlepším Emila.
\cl
\num
„Ty má jedlá, není to tak,
jak to vypadá,
bříza na mně evidentně parazituje!“
„Já kráva bych nejradši se hanbou rozpadla,
své mycelium jsem ti dala a ty s ním skoncuješ!“
\fin
\freev
A až na jaře se z toho sebrala
ňákým houbařem a ten … z ní udělal houbovku.
\cl
\endsong


