\beginsong{Mimorealita}[by={Tomáš Klus}]
\num
\[Em]{Jsou věci,} \[C]který nikdy \[G]nepřekoná\[D]m.
\[Em]{Jsou věci,} \[C]{na který} bud\[G]{u vždycky} \[D]sám.
\[Em]{No tak} se \[C]nezlob, mami, \[G]{je to} jen \[D]zlý zdání. \[Em]Žiju, jsem \[C]zdráv.\[G]  \[D]
Na chvíli \[Em]{bez dobrejch} zpr\[C]áv. \[G]  \[D]
\fin
\chordsoff
\num
Stalo se, stalo a zas může se stát.
Žiju tu poprvý a nesmím se ptát.
Kudy že cesta vede k vysněným cílům, jak neztratit se
a nezranit se.
\fin
\chor
\chordson
Že trošku \[G]bloudím mimo reali\[D]tou,
že lidi v \[Em]běžným světě šedivý \[C]jsou.
Už tak nějak \[G]tuším, že to nebude \[D]snadná úloha,
\[Em]{mít mě} za syna. \[C]\emph{(\rep{2})}
\cl
\freev
\chordson
Lampu \[G]zhasínáš, jdeš \[D]spát
a zítra z\[Em]{ase na} šestou\[C].
A i když \[G]prší, musíš v\[D]stát, já mám tě,
jsi \[Em]{můj kus} nebe. \[C]
\cl
\num
Jsou slova, který prostě nejdou vzít zpět.
Slova, co obrátí naopak svět.
No tak se nezlob, mami, je to jen zlý zdání, žiju jsem zdráv.
Na chvíli bez dobrejch zpráv.
\fin
\num
Sama víš dobře, že to neumím vzdát.
Zas bude líp, prý stačí jen vytrvat.
Člověk si velmi snadno dosáhne na dno, ale musí se rvát.
A pak s hrdostí vstát.
\fin
\chor
Že tak nějak\ldots{} \emph{(\rep{2})}
Lampu zhasínáš\ldots{} \emph{(\rep{2})}
\cl
\endsong


